\documentclass[xcolor=dvipsnames]{beamer}
%\documentclass[xcolor=dvipsnames,handout]{beamer} % Handout mode

\usepackage[utf8]{inputenc} 
\mode<presentation>
{
	\usetheme{Pittsburgh}
	\usecolortheme{crane}
}


\usepackage[ngerman]{babel}
\usepackage[T1]{fontenc}  

\usepackage{tikz-qtree}
\usepackage{tikz}
\usepackage{color}
\usepackage{graphicx}
\usepackage{amsmath,marvosym,amssymb} % Mathesachen
\usepackage[T1]{fontenc} % Ligaturen, richtige Umlaute im PDF
\usepackage{pifont}


% format abbrevations german correctly
\usepackage{xspace}
\newcommand{\vgl}{vgl.\@\xspace} 
\newcommand{\zB}{z.\nolinebreak[4]\hspace{0.125em}\nolinebreak[4]B.\@\xspace}
\newcommand{\bzw}{bzw.\@\xspace}
\newcommand{\dahe}{d.\nolinebreak[4]\hspace{0.125em}h.\nolinebreak[4]\@\xspace}
\newcommand{\etc}{etc.\@\xspace}
\newcommand{\evtl}{evtl.\@\xspace}
\newcommand{\ggf}{ggf.\@\xspace}
\newcommand{\bzgl}{bzgl.\@\xspace}
\newcommand{\so}{s.\nolinebreak[4]\hspace{0.125em}\nolinebreak[4]o.\@\xspace}
\newcommand{\iA}{i.\nolinebreak[4]\hspace{0.125em}\nolinebreak[4]A.\@\xspace}
\newcommand{\sa}{s.\nolinebreak[4]\hspace{0.125em}\nolinebreak[4]a.\@\xspace}
\newcommand{\su}{s.\nolinebreak[4]\hspace{0.125em}\nolinebreak[4]u.\@\xspace}
\newcommand{\ua}{u.\nolinebreak[4]\hspace{0.125em}\nolinebreak[4]a.\@\xspace}
\newcommand{\og}{o.\nolinebreak[4]\hspace{0.125em}\nolinebreak[4]g.\@\xspace}
\newcommand{\oBdA}{o.\nolinebreak[4]\hspace{0.125em}\nolinebreak[4]B.\nolinebreak[4]\hspace{0.125em}d.\nolinebreak[4]\hspace{0.125em}A.\@\xspace}
\newcommand{\OBdA}{O.\nolinebreak[4]\hspace{0.125em}\nolinebreak[4]B.\nolinebreak[4]\hspace{0.125em}d.\nolinebreak[4]\hspace{0.125em}A.\@\xspace}

\usepackage{listings}

\lstset{%
	% Basic design
	basicstyle={\small\ttfamily}, 
	breaklines=true,
	escapeinside={\%*}{*)},
	% Support for German umlauts
	literate=%
	{Ö}{{\"O}}1
	{Ä}{{\"A}}1
	{Ü}{{\"U}}1
	{ß}{{\ss}}1
	{ü}{{\"u}}1
	{ä}{{\"a}}1
	{ö}{{\"o}}1
	{~}{{\textasciitilde}}1
}


% Block that does not breaks over a page
\newenvironment<>{solidblock}[1]{%
	\setbeamercolor{block title}{}%
	\begin{minipage}{\textwidth}
		\begin{block}#2{#1}}
		{\end{block}
	\end{minipage}
}

\newenvironment<>{redblock}[1]{%
	\setbeamercolor{block title}{fg=white,bg=red!75!black}%
	\begin{block}#2{#1}}
	{\end{block}
}

\newenvironment<>{whiteblock}[1]{%
	\setbeamercolor{block title}{fg=black,bg=white!75!black}%
	\begin{block}#2{#1}}
	{\end{block}
}

\newenvironment<>{greenblock}[1]{%
	\setbeamercolor{block title}{fg=black,bg=green!75!black}%
	\begin{block}#2{#1}}
	{\end{block}
}

\newenvironment<>{blackblock}[1]{%
	\setbeamercolor{block title}{fg=white,bg=black!75!black}%
	\begin{block}#2{#1}}
	{\end{block}
}

\newenvironment<>{blueblock}[1]{%
	\setbeamercolor{block title}{fg=white,bg=blue!75!black}%
	\begin{block}#2{#1}}
	{\end{block}
}


\newcommand{\bgpicture}[2]{
	\usebackgroundtemplate{
		\tikz[overlay,remember picture] \node[opacity=#1, at=(current page.center)] {
			%		\hspace{-1.2cm}
			\includegraphics[height=\paperheight]{#2}};
	}
}

\newenvironment{pictureframe}[3][1.0]{%
	\usebackgroundtemplate{%
		\tikz\node[opacity=#1,inner sep=0]{\includegraphics[height=\paperheight,width=\paperwidth]{#2}};
	}
	\begin{frame}\frametitle{#3}
}
{
\end{frame}
\usebackgroundtemplate{}
}

\newenvironment{pictureframebreaks}[3][1.0]{%
\usebackgroundtemplate{
	\tikz\node[opacity=#1,inner sep=0]{\includegraphics[height=\paperheight,width=\paperwidth]{#2}};
}
\begin{frame}[allowframebreaks]\frametitle{#3}
}
{
\end{frame}
\usebackgroundtemplate{}
}

\newenvironment{framebreaks}[1][]{%
\begin{frame}[allowframebreaks]\frametitle{#1}
}
{
\end{frame}
}

\setbeamerfont{frametitle}{size=\large, series=\bf}