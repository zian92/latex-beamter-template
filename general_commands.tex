\RequirePackage[utf8]{inputenc} 

\RequirePackage[ngerman]{babel}

\RequirePackage{color}
\RequirePackage{graphicx}
\RequirePackage{wrapfig}
\RequirePackage{pifont}

 % Math
\RequirePackage{amsmath}
\RequirePackage{marvosym}
\RequirePackage{amssymb}
\RequirePackage{mathtools}

 % Ligaturen, richtige Umlaute im PDF
\RequirePackage[T1]{fontenc}

% Tikz
\RequirePackage{tikz-qtree}
\RequirePackage{tikz}
\usetikzlibrary{arrows,shadows}



% listings
\RequirePackage{listings}

\lstset{%
	% Support for German umlauts
	literate=%
	{Ö}{{\"O}}1
	{Ä}{{\"A}}1
	{Ü}{{\"U}}1
	{ß}{{\ss}}1
	{ü}{{\"u}}1
	{ä}{{\"a}}1
	{ö}{{\"o}}1
	{~}{{\textasciitilde}}1
}


% format abbrevations german correctly
\RequirePackage{xspace}
\newcommand{\vgl}{vgl.\@\xspace} 
\newcommand{\zB}{z.\nolinebreak[4]\hspace{0.125em}\nolinebreak[4]B.\@\xspace}
\newcommand{\bzw}{bzw.\@\xspace}
\newcommand{\dahe}{d.\nolinebreak[4]\hspace{0.125em}h.\nolinebreak[4]\@\xspace}
\newcommand{\etc}{etc.\@\xspace}
\newcommand{\evtl}{evtl.\@\xspace}
\newcommand{\ggf}{ggf.\@\xspace}
\newcommand{\bzgl}{bzgl.\@\xspace}
\newcommand{\so}{s.\nolinebreak[4]\hspace{0.125em}\nolinebreak[4]o.\@\xspace}
\newcommand{\iA}{i.\nolinebreak[4]\hspace{0.125em}\nolinebreak[4]A.\@\xspace}
\newcommand{\sa}{s.\nolinebreak[4]\hspace{0.125em}\nolinebreak[4]a.\@\xspace}
\newcommand{\su}{s.\nolinebreak[4]\hspace{0.125em}\nolinebreak[4]u.\@\xspace}
\newcommand{\ua}{u.\nolinebreak[4]\hspace{0.125em}\nolinebreak[4]a.\@\xspace}
\newcommand{\og}{o.\nolinebreak[4]\hspace{0.125em}\nolinebreak[4]g.\@\xspace}
\newcommand{\oBdA}{o.\nolinebreak[4]\hspace{0.125em}\nolinebreak[4]B.\nolinebreak[4]\hspace{0.125em}d.\nolinebreak[4]\hspace{0.125em}A.\@\xspace}
\newcommand{\OBdA}{O.\nolinebreak[4]\hspace{0.125em}\nolinebreak[4]B.\nolinebreak[4]\hspace{0.125em}d.\nolinebreak[4]\hspace{0.125em}A.\@\xspace}



% custom commands
\newcommand{\inlinecode}{\texttt}

% def for equations
\newcommand{\defeq}{\stackrel{\mathclap{\normalfont\mbox{def}}}{=}}
