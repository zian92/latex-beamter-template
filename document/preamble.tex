\usepackage[utf8]{inputenc}
\usepackage[german]{babel}
\usepackage{fontenc}

\usepackage{setspace}  %Paket um zeilenabstand zu änern
\usepackage{longtable} %mehrseiteige Tabellen
\usepackage{hhline}    %dopplete quer-Linie in Longtable
\usepackage{multirow}
\usepackage{wrapfig}
\usepackage{lipsum}
\usepackage{listings}
\usepackage{enumitem}
\usepackage{pifont}
\usepackage{amsmath}
\usepackage{mathtools}
\usepackage[hidelinks]{hyperref}


% Theorem-Umgebungen
\usepackage{ntheorem}

% diagram 
\usepackage{tikz}
\usetikzlibrary{arrows,shadows}
\usepackage{pgf-umlsd}

\usepackage{graphicx}
\graphicspath{{grafiken/}}
\usepackage{caption}
\usepackage{subcaption}
\usepackage{tikzscale}

\usepackage{color}

\lstset{%
	% Support for German umlauts
	literate=%
	{Ö}{{\"O}}1
	{Ä}{{\"A}}1
	{Ü}{{\"U}}1
	{ß}{{\ss}}1
	{ü}{{\"u}}1
	{ä}{{\"a}}1
	{ö}{{\"o}}1
	{~}{{\textasciitilde}}1
}


\usepackage{beramono}
\newcommand{\inlinecode}{\texttt}

\usepackage{floatrow}
\floatsetup[figure]{capposition=bottom}
\usepackage[tableposition=top]{caption}



% Verweise
\newcommand*{\chref}[1]{Abschnitt \ref{#1} \glqq\nameref{#1}\grqq}
\newcommand*{\absref}[1]{Abbildung \ref{#1} \glqq\nameref{#1}\grqq}
\newcommand*{\tblref}[1]{\ref{#1}}
\newcommand*{\bspref}[1]{\ref{#1}}
\newcommand*{\figref}[1]{\ref{#1}}
\newcommand*{\namedef}[2]{#1 ({#2})}

% Zeichen
\newcommand{\checky}{\ding{51}} % häkchen
\newcommand{\checkn}{\ding{55}} % X

%Hurenkinder, Schusterjungen
% Keine einzelnen Zeilen beim Anfang eines Abschnitts (Schusterjungen)
\clubpenalty = 10000
% Keine einzelnen Zeilen am Ende eines Abschnitts (Hurenkinder)
\widowpenalty = 10000 \displaywidowpenalty = 10000

% itemize bulletpoints
\renewcommand\labelitemi{-}
\renewcommand\labelitemii{-}
\renewcommand\labelitemiii{-}

% Theorem-Optionen %
\theoremstyle{plain}
\theoremseparator{:}
\newtheorem{theorem}{Theorem}[section]
\newtheorem{definition}[theorem]{Definition}
\newtheorem{satz}[theorem]{Satz}
\newtheorem{lemma}[theorem]{Lemma}
\newtheorem{korollar}[theorem]{Korollar}
\newtheorem{proposition}[theorem]{Proposition}
\newtheorem{beweis}[theorem]{Beweis}
\newtheorem{notation}[theorem]{Notation}
\newtheorem{konvention}[theorem]{Konvention}
\newtheorem{bezeichnung}[theorem]{Bezeichnung}
\newtheorem{problem}[theorem]{Problem}
\newtheorem{bemerkung}[theorem]{Bemerkung}
\newtheorem{beobachtung}[theorem]{Beobachtung}
\newtheorem{beispiel}[theorem]{Beispiel}
\newtheorem{frage}[theorem]{Frage}

% Ohne Numerierung
\theoremstyle{nonumberplain}
\renewtheorem{theorem*}{Theorem}
\renewtheorem{satz*}{Satz}
\renewtheorem{lemma*}{Lemma}
\renewtheorem{korollar*}{Korollar}
\renewtheorem{proposition*}{Proposition}
\renewtheorem{definition*}{Definition}
\renewtheorem{bemerkung*}{Bemerkung}
\renewtheorem{beispiel*}{Beispiel}
\renewtheorem{problem*}{Problem}
\renewtheorem{frage*}{Frage}


% Zeilenabstand einstellen %
\renewcommand{\baselinestretch}{1.25}
% Floating-Umgebungen anpassen %
\renewcommand{\topfraction}{0.9}
\renewcommand{\bottomfraction}{0.8}
% Abkuerzungen richtig formatieren %
\usepackage{xspace}
\newcommand{\vgl}{vgl.\@\xspace} 
\newcommand{\zB}{z.\nolinebreak[4]\hspace{0.125em}\nolinebreak[4]B.\@\xspace}
\newcommand{\bzw}{bzw.\@\xspace}
\newcommand{\dahe}{d.\nolinebreak[4]\hspace{0.125em}h.\nolinebreak[4]\@\xspace}
\newcommand{\etc}{etc.\@\xspace}
\newcommand{\evtl}{evtl.\@\xspace}
\newcommand{\ggf}{ggf.\@\xspace}
\newcommand{\bzgl}{bzgl.\@\xspace}
\newcommand{\so}{s.\nolinebreak[4]\hspace{0.125em}\nolinebreak[4]o.\@\xspace}
\newcommand{\iA}{i.\nolinebreak[4]\hspace{0.125em}\nolinebreak[4]A.\@\xspace}
\newcommand{\sa}{s.\nolinebreak[4]\hspace{0.125em}\nolinebreak[4]a.\@\xspace}
\newcommand{\su}{s.\nolinebreak[4]\hspace{0.125em}\nolinebreak[4]u.\@\xspace}
\newcommand{\ua}{u.\nolinebreak[4]\hspace{0.125em}\nolinebreak[4]a.\@\xspace}
\newcommand{\og}{o.\nolinebreak[4]\hspace{0.125em}\nolinebreak[4]g.\@\xspace}
\newcommand{\oBdA}{o.\nolinebreak[4]\hspace{0.125em}\nolinebreak[4]B.\nolinebreak[4]\hspace{0.125em}d.\nolinebreak[4]\hspace{0.125em}A.\@\xspace}
\newcommand{\OBdA}{O.\nolinebreak[4]\hspace{0.125em}\nolinebreak[4]B.\nolinebreak[4]\hspace{0.125em}d.\nolinebreak[4]\hspace{0.125em}A.\@\xspace}

% Leere Seite ohne Seitennummer, naechste Seite rechts
\newcommand{\blankpage}{
	\clearpage{\pagestyle{empty}\cleardoublepage}
}

% itemize/enumerate mit wenig Abstand zwischen den Items
%
\newenvironment{itemizesmall}
{
	\begin{itemize}[topsep=0pt,itemsep=-1ex,partopsep=1ex,parsep=1ex]
	}{
\end{itemize}
}

\newenvironment{enumeratesmall}
{
	\begin{enumerate}[topsep=0pt,itemsep=-1ex,partopsep=1ex,parsep=1ex]
	}{
\end{enumerate}
}



% Definitionen für Gleichungen
\newcommand{\defeq}{\stackrel{\mathclap{\normalfont\mbox{def}}}{=}}